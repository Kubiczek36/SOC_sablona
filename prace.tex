% ŠABLONA PRO PSANÍ SOČ
%%%%%%%%%%%%%%%%%%%%%%%%%
% Autor: Jakub Dokulil (kubadokulil99@gmail.com)
% Tato šablona byla vytvořena tak, aby pomocí ní mohli v systému LaTeX soutěžící sázet své práce a zároveň odpovídala požadavkům na formátování vyplývajícím z wordové šablony umístěné na webu soc.cz.
%
\documentclass[12pt, a4paper,
  %oneside,      %% -- odkomentujte, pokud chcete svou práci mít pouze jednostrannou, mezera pro hřbet pak automaticky bude pouze na levé straně
 twoside,        %% -- pro oboustranné práce, mezera pro hřbet následně střídá strany.
 openright
]{report}

%% Nutné balíčky a nastavení
%%%%%%%%%%%%%%%%%%%%%%%%%%%%

\title{Moc důležitý a zajímavý název v češtině} %% -- Název tvé práce
\author{Dummy Author} %% -- tvé jméno
\date{3020} %% -- rok, kdy píšeš SOČku

\usepackage[top=2.5cm, bottom=2.5cm, left=3.5cm, right=1.5cm]{geometry} %% nastaví okraje, left -- vnitřní okraj, right -- vnější okraj

\usepackage[czech]{babel} %% balík babel pro sazbu v češtině
\usepackage[utf8]{inputenc} %% balíky pro kódování textu
\usepackage[T1]{fontenc}
\usepackage{cmap} %% balíček zajíšťující, že vytvořené PDF bude prohledávatelné a kopírovatelné

\usepackage{graphicx} %% balík pro vkládání obrázků

\usepackage{subcaption} %% balíček pro vkládání podobrázků

\usepackage{hyperref}

\linespread{1.15} %% řádkování

\usepackage[pagestyles]{titlesec} %% balíček pro úpravu stylu kapitol a sekcí
\titleformat{\chapter}[block]{\scshape\bfseries\LARGE}{\thechapter}{10pt}{\vspace{0pt}}[\vspace{-22pt}]
\titleformat{\section}[block]{\scshape\bfseries\Large}{\thesection}{10pt}{\vspace{0pt}}
\titleformat{\subsection}[block]{\bfseries\large}{\thesubsection}{10pt}{\vspace{0pt}}

\setcounter{secnumdepth}{2}
\setcounter{tocdepth}{1}
\usepackage{fancyhdr}
\pagestyle{fancy}
\renewcommand{\headrulewidth}{1pt}

\usepackage{booktabs}

\usepackage{url}

%% Balíčky co se můžou hodit :) 
%%%%%%%%%%%%%%%%%%%%%%%%%%%%%%%

\usepackage{pdfpages} %% Balíček umožňující vkládat stránky z PDF souborů, balíček je nutný pokud chcete 


\usepackage{upgreek} %% Balíček pro sazbu stojatých řeckých písmen, například u jednotky mikrometr. Například stojaté mí: \upmu, stojaté pí: \uppi

\usepackage{amsmath}    %% Balíčky amsmath a amsfonts 
\usepackage{amsfonts}   %% pro sazbu matematických symbolů
\usepackage{esint}     %% pro sazbu různých integrálů (např \oiint)
\usepackage{mathrsfs}

\usepackage{listings} %% balíček pro sazbu zdrojových kódů

\usepackage{color} %red, green, blue, yellow, cyan, magenta, black, white
\definecolor{mygreen}{RGB}{28,172,0} % color values Red, Green, Blue
\definecolor{mylilas}{RGB}{170,55,241}

\lstset{language=Python, %% sem napiš jaký programovací jazyk používáš, úplný seznam viz: https://en.wikibooks.org/wiki/LaTeX/Source_Code_Listings
    %basicstyle=\color{red},
    breaklines=true,%
    basicstyle=\ttfamily,
%    morekeywords={matlab2tikz},
    keywordstyle=\color{blue},%
    morekeywords=[2]{1}, keywordstyle=[2]{\color{black}},
    identifierstyle=\color{black},%
    stringstyle=\color{mylilas},
    commentstyle=\color{mygreen},%
    showstringspaces=false,%without this there will be a symbol in the places where there is a space
    numbers=left,%
    numberstyle={\tiny \color{black}},% size of the numbers
    numbersep=9pt, % this defines how far the numbers are from the text
    emph=[1]{for,end,break},emphstyle=[1]\color{red}, %some words to emphasise
    %emph=[2]{word1,word2}, emphstyle=[2]{style},    
}


%% Bordel pro práci - později se smázne :) 
%%%%%%%%%%%%%%%%%%%

\usepackage{lipsum}

%% Začátek dokumentu
%%%%%%%%%%%%%%%%%%%%
\begin{document}

\pagestyle{empty}
\pagenumbering{Roman}

\begin{titlepage}
    \bfseries{ %%% písmo na stránce je tučně
        \begin{center}
            \LARGE{STŘEDOŠKOLSKÁ ODBORNÁ ČINNOST}

            \vspace{14pt}
            \large{ %%%%
                Obor č. 21: Futuramologie %% -- napiš číslo a název tvého oboru
            } %%%%

            \vspace{0.4 \textheight}

            \LARGE{ %%%%
                Pojednání o vražedném Santovi
            }%%%%

            \vspace{0.4\textheight}
        \end{center}
        
        \noindent\Large{Philip J. Fry}  %% vyplň své jméno

        \noindent\Large{Galaktický kraj \hspace{\stretch{1}}  Nový New York, 3020} %% vyplň oficiální název kraje, město a rok
        
            
    } %%%
\end{titlepage}

\cleardoublepage

%% Úvodní stránka s informacemi
{\bfseries %%% písmo na stránce je tučně
    \begin{center}
        \LARGE{STŘEDOŠKOLSKÁ ODBORNÁ ČINNOST}

        \vspace{14pt}
        {\large %%%%
            Obor č. 21: Futuramologie %% -- napiš číslo a název tvého oboru
        } %%%%

        \vspace{0.3 \textheight}

        \LARGE{ %%%%
        Pojednání o vražedném Santovi
        }

        \LARGE{ %%%%
        Tract about killing Santa
        }%%%%

        \vspace{0.24\textheight}
    \end{center}  
}%%%
{\Large %%%
    \noindent\textbf{Jméno:} Philip J. Fry\\
    \textbf{Škola:} Mars University\\
    \textbf{Kraj:} Galaktický kraj\\
    \textbf{Konzultant:} prof. Hubert J. Farnsworth\\
} %%%

\noindent Nový New York, 3020

\cleardoublepage

\noindent{\Large{\bfseries{Prohlášení}}}  %% uprav si koncovky podle toho na jaký rod se cítíš, vypadá to pak lépe :) 

\noindent Prohlašuji, že jsem svou práci SOČ vypracoval/a samostatně a použil/a jsem pouze prameny a literaturu uvedené v seznamu bibliografických záznamů.

\noindent Prohlašuji, že tištěná verze a elektronická verze soutěžní práce SOČ jsou shodné. 

\noindent Nemám závažný důvod proti zpřístupňování této práce v souladu se zákonem č. 121/2000 Sb., o právu autorském, o právech souvisejících s právem autorským a o změně některých zákonů (autorský zákon) ve znění pozdějších předpisů. 

\vspace{24 pt}

\noindent V Novém New Yourku dne 9. září 3020 \dotfill{}\hspace{\stretch{0.5}} 

\hspace{8cm} Philip J. Fry

\cleardoublepage

\vspace*{0.8\textheight}
\noindent{\Large{\bfseries{Poděkování}}}

\noindent
Chtěl bych poděkovat mému školiteli, prof. Farnsworthovi, za jeho úžasné tipy, triky a připomínky, bez kterých by nevznikla tato práce. Dále bych chtěl poděkovat mé rodině a přítelkyni, za to, že mě dostatečně zásobili kávou.

\cleardoublepage

\tableofcontents

\pagenumbering{arabic}
\pagestyle{fancy}
\setcounter{page}{1}

\chapter*{Úvod}
Ahoj,
a vítám tě u této šablony pro psaní SOČky v \LaTeX u. Moc mne těší, že sis vybral právě typografický systém \LaTeX pro psaní své práce, jelikož jsem přesvědčen, že s jeho pomocí dosáhneš nejlepšího výsledku. Tvá práce pak bude vypadat, krásně, elegantně a profesionálně a tím snáz uděláš dobrý dojem na porotu. Pokud se s \LaTeX em teprve učíš, tak nevěš hlavu, i na tebe jsem myslel. V následujících kapitolách této šablony najdeš tipy a triky, jak psát práci a jak vytáhnout z \LaTeX u to nejlepší (a že toho umí). Zároveň je dobré sledovat komentáře v zdrojovém kódu, díky nim snáz pochopíš, k čemu je jaký příkaz. V případě kdyby něco nesedělo, nebo si na mě měl jakýkoli dotaz, tak se na mě můžeš jednak obrátit na GitHubu \cite{sablonaSOC}, kde je tato šablona uložena  a nebo přímo na můj mail \url{kubadokulil99@gmail.com}

Ale teď už hurá na psaní!

\chapter{Tipy k psaní}

Jak už jsem psal výše \LaTeX je dosti komplexní systém, který umožňuje psát velmi rozsáhlé text. Jeho autor Donald Knuth ho stvořil, aby mohl vydat jeho učebnici \emph{The Art of Computer Programming} a dodnes se je využíván pro sazbu vysokoškolských skript a učebnic. V této kapitole najdeš ukázky různých funkcí a balíčků \LaTeX u od těch nejzákladnějších až po složitější. Neznamená to nutně, že všechny musíš použít, ale když potřebuješ pomoct, tak je dobré mít oporu. 

Pokud s \LaTeX em úplně zařínáš tak ti můžu doporučit přiručku \emph{Ne příliš stručný úvod do systému \LaTeX2e}~\cite{LaTeXprirucka}. Případně spoustu užitečných informací nalezneš na Wikibooks~\cite{wikibooksLaTeX}.


\section[Základy]{Základy: Text, obrázky, tabulky a citace}
Psaní textu v tomto není žádná věda, stačí psát normálně do zdrojového souboru. Pokud bys chtěl psát obrážky či číslovaný seznam, pak můžeš použít prostředí \texttt{itemize} či \texttt{enumerate}. Pokud často je důležité používat nezlomitelnou mezeru. Tu uděláš pomocí \verb|~| (tildy).

U obrázků je dobré používat vektorové formáty, pokud to jde. \LaTeX se nejvíc kamarádí s formátem PDF. Do známého PDFka lze z jiných vektorových formátů (ať už SVG či ESP) obrázky přenést snadno pomocí grafických programů, jako je třeba inkscape. \LaTeX si rozhodně poradí i s tradičními formáty PNG a JPG, avšak tyto obrázky mohou zabírat více prostoru a kolikrát

\begin{figure}[h]
    \centering
    \includegraphics[width=0.6\textwidth]{imgs/soc-logo.jpg}
    \caption{Logo SOČky}
    \label{fig:logoSOC}
\end{figure}

\begin{table}
    \caption{Tato tabulka slouží jako ukázka toho, jak mohou tabulky vypadat.}
    \begin{center}
        \begin{tabular}{lll}
            \toprule
            záhlaví& této & tabulky\\
            \midrule
            obsah&tabulky& už\\
            není & oddělený &čarami\\
            \bottomrule
        \end{tabular}
    \end{center}
\end{table}

\subsection{Some even more specific section maybe with even longer and longer name}
\lipsum[1-2]
\begin{equation}
    \oint B \cdot d \vec{s} = \upmu_0 I
\end{equation}

\lstinputlisting[firstline=3, lastline=7]{code/example.py}

\lipsum[1-3]

\chapter[Třeba to ale nějak zkrátím]{A co když budu mít nějakou fakt moc dlouhou kapitolu -- teda název -- třeba i s rovnicí $E = mc^2$}

Co dále udělat

\begin{itemize}
    \item přidat odkaz na ne příliš stručný úvod do systému \LaTeX2e
    \item odkaz na wikibooks
    \item tex overflow
    \item obrázky, vektor - rastr
    \item 
\end{itemize}


%% literatura
\begin{thebibliography}{99}
    \bibitem{sablonaSOC} DOKULIL Jakub. \textit{Šablona pro psaní SOČ v programu \LaTeX} [Online]. Brno, 2020 [cit. 2020-08-24]. Dostupné z: \url{https://github.com/Kubiczek36/SOC_sablona}
    \bibitem{LaTeXprirucka}OETIKER, Tobias, Hubert PARTL, Irene HYNA, Elisabeth SCHEGL, Michal KOČER a Pavel SÝKORA. \textit{Na příliš stručný úvod do systému LaTeX2e} [online]. 1998 [cit. 2020-08-24]. Dostupné z: \url{https://www.jaroska.cz/elearning/informatika/typografie/lshort2e-cz.pdf}
    \bibitem{wikibooksLaTeX}\textit{Wikibooks: LaTeX} [online]. San Francisco (CA): Wikimedia Foundation, 2001- [cit. 2020-08-24]. Dostupné z: \url{https://en.wikibooks.org/wiki/LaTeX}
    \bibitem{Born2019}
    BORN, Max a Emil WOLF. \textit{Principles of optics: electromagnetic theory of propagation, interference and diffraction of light}. 7th (expanded) edition. Reprinted wirth corrections 2002. 15th printing 2019. Cambridge: Cambridge University Press, 2019. ISBN 978-0-521-64222-4.
\end{thebibliography}

%% obrázky 
\listoffigures

%% tabulky
\listoftables

\appendix %% začínají přílohy

\titleformat{\chapter}[block]{\scshape\bfseries\LARGE}{Příloha \thechapter}{10pt}{\vspace{0pt}}[\vspace{-22pt}] %% nastavení nadpisu


\chapter{%Příloha A 
Spot diagramy a další }

\lstinputlisting{code/example.py}

\end{document}