% ŠABLONA PRO PSANÍ SOČ
%%%%%%%%%%%%%%%%%%%%%%%%%
% Autor: Jakub Dokulil
% Tato šablona byla vytvořena tak, aby pomocí ní mohli v systému LaTeX soutěžící sázet své práce a zároveň odpovídala požadavkům na formátování vyplývajícím z wordové šablony umístěné na webu soc.cz.
%
\documentclass[12pt, a4paper,s
  %oneside,      %% -- odkomentujte, pokud chcete svou práci mít pouze jednostrannou, mezera pro hřbet pak automaticky bude pouze na levé straně
 twoside,        %% -- pro oboustranné práce, mezera pro hřbet následně střídá strany.
 openright
]{report}

%% Nutné balíčky a nastavení
%%%%%%%%%%%%%%%%%%%%%%%%%%%%

\title{Moc důležitý a zajímavý název v češtině} %% -- Název tvé práce
\author{Dummy Author} %% -- tvé jméno
\date{2020} %% -- rok, kdy píšeš SOČku

\usepackage[top=2.5cm, bottom=2.5cm, left=3.5cm, right=1.5cm]{geometry} %% nastaví okraje, left -- vnitřní okraj, right -- vnější okraj

\usepackage[czech]{babel} %% balík babel pro sazbu v češtině
\usepackage[utf8]{inputenc} %% balíky pro kódování textu
\usepackage[T1]{fontenc}
\usepackage{cmap} %% balíček zajíšťující, že vytvořené PDF bude prohledávatelné a kopírovatelné

\usepackage{graphicx} %% balík pro vkládání obrázků

\usepackage{subcaption} %% balíček pro vkládání podobrázků

\linespread{1.15} %% řádkování

\usepackage[pagestyles]{titlesec} %% balíček pro úpravu stylu kapitol a sekcí
\titleformat{\chapter}[block]{\scshape\bfseries\LARGE}{\thechapter}{10pt}{\vspace{0pt}}[\vspace{-22pt}]
\titleformat{\section}[block]{\scshape\bfseries\Large}{\thesection}{10pt}{\vspace{0pt}}
\titleformat{\subsection}[block]{\bfseries\large}{\thesection}{10pt}{\vspace{0pt}}

\setcounter{secnumdepth}{2}
\setcounter{tocdepth}{1}
\usepackage{fancyhdr}
\pagestyle{fancy}
\renewcommand{\headrulewidth}{1pt}

%% Balíčky co se můžou hodit :) 
%%%%%%%%%%%%%%%%%%%%%%%%%%%%%%%

\usepackage{pdfpages} %% Balíček umožňující vkládat stránky z PDF souborů, balíček je nutný pokud chcete 

\usepackage{upgreek} %% Balíček pro sazbu stojatých řeckých písmen, například u jednotky mikrometr. Například stojaté mí: \upmu, stojaté pí: \uppi

\usepackage{amsmath}    %% Balíčky amsmath a amsfonts 
\usepackage{amsfonts}   %% pro sazbu matematických symbolů
\usepackage{esint}     %% pro sazbu různých integrálů (např \oiint)
\usepackage{mathrsfs}

%% Bordel pro práci - později se smázne :) 
%%%%%%%%%%%%%%%%%%%

\usepackage{lipsum}

%% Začátek dokumentu
%%%%%%%%%%%%%%%%%%%%
\begin{document}

\pagestyle{empty}

\begin{titlepage}
    \bfseries{ %%% písmo na stránce je tučně
        \begin{center}
            \LARGE{STŘEDOŠKOLSKÁ ODBORNÁ ČINNOST}

            \vspace{14pt}
            \large{ %%%%
                Obor č. 21: Futuramologie %% -- napiš číslo a název tvého oboru
            } %%%%

            \vspace{0.4 \textheight}

            \LARGE{ %%%%
                Pojednání o vražedném Santovi
            }%%%%

            \vspace{0.4\textheight}
        \end{center}
        
        \noindent\Large{Philip J. Fry}  %% vyplň své jméno

        \noindent\Large{Galaktický kraj \hspace{\stretch{1}}  Nový New York, 3020} %% vyplň oficiální název kraje, město a rok
        
            
    } %%%
\end{titlepage}

\cleardoublepage

%% Úvodní stránka s informacemi
{\bfseries %%% písmo na stránce je tučně
    \begin{center}
        \LARGE{STŘEDOŠKOLSKÁ ODBORNÁ ČINNOST}

        \vspace{14pt}
        {\large %%%%
            Obor č. 21: Futuramologie %% -- napiš číslo a název tvého oboru
        } %%%%

        \vspace{0.3 \textheight}

        \LARGE{ %%%%
        Pojednání o vražedném Santovi
        }

        \LARGE{ %%%%
        Tract about killing Santa
        }%%%%

        \vspace{0.24\textheight}
    \end{center}  
}%%%
{\Large %%%
    \noindent\textbf{Jméno:} Hubert J. Farnsworth\\
    \textbf{Škola:} Mars University\\
    \textbf{Kraj:} Galaktický kraj\\
    \textbf{Konzultant:} Hubert J. Farnsworth\\
} %%%

\noindent Nový New York, 3020

\cleardoublepage

{\Large{\bfseries{Prohlášení}}} %% uprav si koncovky podle toho na jaký rod se cítíš, vypadá to pak lépe :) 

\noindent Prohlašuji, že jsem svou práci SOČ vypracoval/a samostatně a použil/a jsem pouze prameny a literaturu uvedené v seznamu bibliografických záznamů.

\noindent Prohlašuji, že tištěná verze a elektronická verze soutěžní práce SOČ jsou shodné. 

\noindent Nemám závažný důvod proti zpřístupňování této práce v souladu se zákonem č. 121/2000 Sb., o právu autorském, o právech souvisejících s právem autorským a o změně některých zákonů (autorský zákon) ve znění pozdějších předpisů. 

\vspace{24 pt}

\noindent V Novém New Yourku dne 9. září 3020 \dotfill{}\hspace{\stretch{0.5}} 

\hspace{8cm} Philip J. Fry


\tableofcontents

\chapter{Some chaper}
\lipsum[1-2]

\section[Třeba taky udělám zkratku]{Some more specific section, with a quite very long name}
\lipsum[1-2]

\begin{figure}[h]
    \centering
    \includegraphics[width=0.6\textwidth]{imgs/soc-logo.jpg}
    \caption{Logo SOČky}
    \label{fig:logoSOC}
\end{figure}

\subsection{Some even more specific section maybe with even longer and longer name}
\lipsum[1-2]
\begin{equation}
    \oint B \cdot d \vec{s} = \upmu_0 I
\end{equation}
\lipsum[1-3]

\chapter[Třeba to ale nějak zkrátím]{A co když budu mít nějakou fakt moc kurva dlouhou kapitolu -- teda název -- třeba i s rovnicí $E = mc^2$}

\end{document}